\documentclass[a4paper, 11pt, onecolumn]{article}

% PACKAGES

\usepackage{float}
\usepackage[english]{babel}
\usepackage[singlespacing]{setspace} % 1,5 Zeilenabstand: onehalfspacing
\usepackage{pgfplots}  % Nice plots
\usepackage{minted}  % Code highlighting
\usepackage{xspace}  % Add trailing space if neccesary
\usepackage{todonotes}
\usepackage[colorlinks=true,
            allcolors=orange,
            breaklinks]{hyperref}

% \setlength{\marginparwidth}{2cm}

% COMMANDS

% Show a code sample with the resulting plot side by side
% Works well for code that has a max of 42 characters per line
\newcommand{\plotsample}[6][12cm]{
\begin{figure*}[h]
    
    \vspace{#2}
    \hspace{-1cm}
    \begin{minipage}[c][#1][t]{#3\textwidth} 
    \inputminted{python}{#4.py}
    \end{minipage}
    \hspace{0.5cm}
    \begin{minipage}[c][#1][t]{0.6\textwidth}
    \IfFileExists{#4.pgf}{
        \input{#4.pgf}
    }{
        \includegraphics[width=\linewidth]{#4.png}
    }
    \end{minipage}

\caption{#5}
\label{smpl:#6}
\end{figure*}
}

\newcommand{\mpl}{\texttt{matplotlib}\xspace}
\newcommand{\pl}[1]{\mbox{\texttt{#1}\xspace}}
\newcommand{\latex}{\LaTeX\xspace}

\newcommand{\fref}[2]{\href{#1}{#2}\footnote{\url{#1}}}
% reference helpers
\newcommand{\baseref}[3]{\mbox{#1 \ref{#2:#3}\xspace}}
\newcommand{\fig}[1]{\baseref{figure}{fig}{#1}}
\renewcommand{\sec}[1]{\baseref{section}{sec}{#1}}
\newcommand{\lis}[1]{\baseref{listing}{l}{#1}}
\newcommand{\lin}[1]{\baseref{line}{li}{#1}}

\title{matplotlib et al.}
\author{Johannes Haux}
\date{\today}

%DOCUMENT
\begin{document}

\maketitle


Presenting data has always been a very important topic in research. 
While quantitative analyses can give precise information about data many
use cases exist, where it is far easier to convey information using graphical
representations 
\cite{plotsarecool}.

Today a large number of tools exist to generate plots from data.  Commercial
solutions like \pl{Mathematica} \cite{mate}, \pl{MATLAB} \cite{matl} or
\pl{Origin} \cite{orig} incorporate data analysis software together with
plotting tools, allowing to have a complete pipeline from reading and analyzing
the data to generating a finished plot.

Free and open source libraries that offer similar functionality are \pl{d3.js}
\cite{d3js}, which is the base for \pl{plotly.js} \cite{plotly}, or \mpl
\cite{mpl}, which in turn is used as backend in \pl{bokeh} \cite{bokeh} and
\pl{seaborn}, \cite{seaborn}.

\pl{d3.js} is a very low level \pl{JavaScript} library which offers very
fine grained possibilities to generate interactive data visualizations.

\pl{plotly.js}, which is built on top of it offers only a subset of plot types,
but with more convenience functions, making it easier to generate plots. 
Plotly can be used as JavaScript library, but also bindings for \pl{python}, 
\pl{MATLAB} or \pl{R} exist.

The \mpl is a python library, originally written by John Hunter to make 
visualizations of the brain \cite{john}. It later became the most widely used
python plotting library.

Built on top of it are \pl{bokeh} and \pl{seaborn}, which offer higher level
abstractions and a set of predefined styles. Additionally \pl{bokeh} also
makes plots interactive, that can be integrated in web pages.


\section{This paper}

When it comes to presenting data, \mpl is a very interesting
tool, that combines a lot of positive features:
It is 
free,
well documented with a lot of examples and
customizable to a very fine grained level.

This document accompanies a talk given in the seminar "Advanced Tools" during
the summer semester 2017. Here I will take a look at some basic concepts of
\mpl and then demonstrate some features using an example plot.

For more interactive examples I recommend to take a look the jupyter notebooks
from the talk, which can be found on
\fref{https://github.com/jhaux/matplotlib\_et\_al}{github}. 

\section{A quick Plot}

\begin{figure}[h!]
\centering
\input{axis_labels.pgf}
\caption{Plot generated with the code shown in \lis{al}}
\label{fig:al}
\end{figure}

\begin{listing}[h!]
\inputminted{python}{axis_labels.py}
\caption{Code for \fig{al}.}
\label{l:al}
\end{listing}

To start of we take a look at how easy it is to quickly generate plots with
the \mpl.
In \fig{al} you can find a plot made with the script shown
in \lis{al}.

Given an artificially created dataset \pl{data} we want to take a look at its
histogram.
To do this only one line of code is used to generate the plot: 
\pl{plt.hist(data)}
Using the commands \pl{plt.[x,y]label} one can quickly add labels to the
axis to make the plot easier to interpret.

Finally the plot is saved to using the \pl{plt.savefig(...)} command and shown
to the user with the \pl{plt.show()} command.

To ensure that the axis labels are visible when the plot is saved the method
\pl{f.tight\_layout()} is called on the figure object containing the plot.

This shows just how easy it is to get a visual feedback of how your data looks
like with only a small number of lines of code! 

But when thinking about publishing, one wants to have more intricate
customizability. This is absolutely possible using the \mpl, as can be seen for
example in \fig{gs}.  To understand what is happening there, we will
first take a look at what is inside the plots generated with the \mpl.


\section{Components of a Plot}

A \mpl plot consists of 4 main components:
\begin{description}
\item[Figure]
The figure manages all axes in a plot and contains information on the size of
the plot. It can be instantiated by calling \pl{plt.figure()}. 
The current figure can be obtained by calling \pl{plt.gcf()}.

\item[Axes]
The axes is where the actual plot lives. Here the data is drawn
as specified in the plot calls. To instantiate an axes given a figure \pl{f}
one can use \pl{f.add\_subplot(...)}. To generate both figure and axes it
is encouraged to use \pl{f, ax = plt.subplots(rows, cols)}.
The current axes can be obtained by calling \pl{plt.gca()}.

\item[Axis]
Each axes contains a set of axis, which manage the data range that is to be
plotted as well as all ticks and tick labels.

\item[Artist]
All components of a plot are drawn by artists when the plot is rendered.
E.g. a figure is also an artist, as well as axes and axis.
\end{description}

The \mpl has a notion of a current figure and axis. This allows
the user to make several plotting calls without explicitly specifying where
to draw the plot. As said above, these can be accessed by calling 
\pl{plt.gcf()} and \pl{plt.gca()} for current figure and axes respectively.


\section{Working with \mpl}

Before taking a look at a more complex example, let us first have a look at
ways of helping yourself if the many possibilities the \mpl offers become overwhelming. 

\begin{description}
\item[Examples]
If one is unsure what kind of plot one wants to make or has a rough idea, what
it should look like, the \fref{https://matplotlib.org/examples/}{examples} page
of the \mpl documentation is very helpful.
\item[Documentation]
When in doubt how to use a certain object or method, the
\mpl \fref{https://matplotlib.org/contents.html}{documentation} covers it! 
\item[StackOverflow]
If the above two ways do not help, it might be useful to note, that there is a
very active user community, helping each other out, e.g. on
\fref{https://stackoverflow.com/questions/tagged/matplotlib}{stackoverflow}.
Most solutions to most problems can be found there.
\end{description}



\section{A more complex Plot}

\label{sec:plt}

\begin{figure}[ht!]
\centering
\input{grid_spec.pgf}
\caption{Complex plot example showing different plot types and ways to
         structure the appearance. It is discussed in detail in \sec{plt}.}
\label{fig:gs}
\end{figure}

With the knowledge of the \mpl components and how to find documentation, let us
take a look at a more complex plot example.


In the following, whenever line numbers are referenced, these are taken from
\sec{gs}, unless otherwise specified.

In \fig{gs} one can find the plot that is generated by the code
found in \sec{gs}. Again artificially generated data is used.

When comparing the code shown in \lis{al} and \sec{gs}
respectively, one can first of all notice several differences in how the plots
are created.

\subsection{Coding Style}
The code shown in \fig{al} is written in the "pyplot" coding style,
also called "MATLAB" coding style, as it tries to emulate the plotting commands
used in MATLAB.

This coding style can be used to quickly create simple plots, yet it can
grow unreadable very fast as the plot becomes more and more complex.

The alternative to this coding style is the object oriented approach, which 
uses explicit handles for figures and axes objects. It has the great advantage,
that it make s code easier to read and allows access to more methods, that 
would be harder to use when using pure pyplot coding.

In \lin{fig} 


\subsection{Multiple Axes}

The simple plot consisted only of one axes and figure, that were both
implicitly created with the \pl{plt.hist()} call. In the second example we find
several axes in the figure, which are created explicitly.

There are several ways to draw multiple axes in one figure. When wanting to
draw a regular grid of plots the easiest method would be
\mint{python}|fig, AX = plt.subplots(nrows, ncols)|
which returns the figure and a list of all axes. The arguments \pl{nrows} and 
\pl{ncols} define the numbers of rows and columns of axes in the figure.

To get more control over the way the axes are drawn one can use the
\pl{GridSpec} object as is done in the example plot. It also defines a grid,
but in contrast to the approach above one can also define axes that span
several rows or columns or both, by passing slices of the object to
\pl{figure.add\_subplot()}. This is done for example in \lin{gsslice}.
Additionally one can also specify the height or width ratios of the axes using
the \pl{height\_ratios} or \pl{width\_ratios} keyword argument. The latter can
be seen in \lin{gswr}. Of course the customizability does not end here,
more information on the
\fref{https://matplotlib.org/users/gridspec.html}{GridSpec} object can be found
in the \mpl documentation.


\subsection{Plot types}

We can also see a new plot type in \fig{gs}: The line plot, generated
with the \pl{ax.plot()} method (see \lin{plot}).

There are very many different plot types \mpl supports, a look at the
\fref{https://matplotlib.org/api/axes\_api.html\#plotting}{documentation} can
give an idea of how these look like.

Also note that when plotting the histogram in \lin{hist} and below, some
keyword arguments are passed to draw a black edge around the bars and make them
translucent, such that the overlapping parts are all visible. Most plotting
methods support these and similar arguments, like \pl{linestyle}, \pl{color},
\pl{marker} or \pl{markersize}, which allow to explicitly specify the way the
plots are drawn.

Calling several plotting methods on one axis in succession without specifying
the color explicitly will not result in multiple plots of the same color, but
as can be seen in \fig{gs}, in plots of different colors. This allows to easily
stack plots in one axes without loosing the overview.

\subsection{Legends}

To know which plot corresponds to which data one can pass the \pl{label} 
keyword to each plotting method and call the \pl{ax.legend()} method on the
axes of interest. The draws a legend with the specified labels in a place on
the axes that covers as little of the plot as possible. This can be seen in
\lin{leg}. The position of the legend can also be explicitly set by 
passing a string (e.g. \pl{'upper left'}, \pl{'center'}) or an int (e.g.
\pl{2}, \pl{10} respectively).

Calling \pl{legend()} without arguments will only take labels from plotting
calls on this axes into account. If one would like to make a legend inside a
different axes than the one where the data is plotted, one has to pass a list
of handles and a list of corresponding names to this method. The handles are
returned by the plotting methods.

Again a look at the \fref{https://matplotlib.org/api/\_as\_gen/matplotlib.axes.Axes.legend.html\#matplotlib.axes.Axes.legend}{documentation}
can be very helpful.


\subsection{Text and Annotation}

The most glaring difference between the simple and the complex plot are, that 
the in the latter the fonts are set in a latex like serif font. This is
accomplished by telling the \mpl to update some default parameters stored
in the \pl{rcParams} dictionary. To simply get use latex text, one has to set
the \pl{'text.usetex'} parameter to True and to use serif fonts, one has to
choose the \pl{'font.family'} accordingly. All text will now look like in a
\latex document.

But this is not all we can do: If one would like to draw a \latex formula in
the plot this is now possible. In the example this is done in \lin{ann}
using the \fref{https://matplotlib.org/users/annotations.html}{\pl{annotate()}}
method, but one could also use the
\fref{https://matplotlib.org/api/pyplot\_api.html\#matplotlib.pyplot.text}{\pl{text()}} method.

Please note that the string containing the text is a raw string. Thus no 
characters are escaped, which helps a lot when using expressions like
\pl{\\frac\{\}\{\}}, that would otherwise result in escaping \pl{\\f} to a
formfeed or page break, something latex does not understand.

While \pl{text} simply draws the text where we want it, \pl{annotate} also
allows to draw an arrow between text and a point \pl{xy}, we have to specify.
The properties of the arrow are controlled using the \pl{arrow\_props}
dictionary. Here we chose to draw an arc with a simple arrowhead on the front,
but many other possibilities exist.

The coordinates that are specified are also interesting to look at. When
drawing text the default behavior of \mpl is to interpret the coordinates like
data points. In our example this is the case for the \pl{xy} coordinates where
the arrow points. As the text is supposed to be place in the top left corner,
the coordinates \pl{xytext} here are chosen to be a fraction of the axes as
specified by passing the \pl{textcoords} argument. (0, 0) corresponds to the
bottom left corner of the axes, (1, 1) to the top right. Values outside the
$[0, 1]^2$ are outside of the axes.

To ensure that the text is not drawn over the borders of the plot we can
specify its alignment with the \pl{va} and \pl{ha} keywords for vertical and
horizontal alignment respectively. In our example they are set such that the 
top right of the bounding box of the text is at the coordinates specified for 
the text.

Another example where text is modified can be found in \lin{lbc}.
Sometimes it can be useful to explicitly position the labels of an axis. In
our example the y labels of the top left and bottom axes were not aligned, so
they were moved using the \pl{ax.yaxis.set\_label\_coords()} method. Its
arguments are the x and y coordinates of the label as axes fraction at which
the text should be placed, just like above in the \pl{annotate} example.


\subsection{Spans}

Sometimes it can be necessary to draw geometry spanning the whole plot.
In \lin{hline} of the example a dashed horizontal line is drawn and in
\lin{vspan} we can find an example of how to draw an area inside the plot.

The coordinates that are input to these methods are expected to be in data
space, making these methods very useful when pointing out special
characteristics of the data. In the case of the example it is the mean of
the \pl{Y} dataset and the difference in \pl{X} of one period of the \pl{Z}
dataset.


\subsection{Exporting}

When we are done with tweaking the plot it is time to save it to a file, that
can later be reused. This is simply done by calling the \pl{f.save()} method 
on the figure object. The only thing that needs to be specified is name of the
file with the file ending corresponding the desired format. This could for 
example be \pl{.png}, \pl{.jpg}, \pl{.svg} or \pl{.pgf}.

The latter is of special interest to \latex users, as it saves the plot as a
tikz picture, that then can be placed inside a \latex document using the
command \pl{\\input\{<plot>.pgf\}}

The dimensions of the plot are those specified when instantiating the figure
using the \pl{figsize} argument. Please note that the unit of these values is
inches. 

It is also important to note, that the size of text in the plots corresponds
to the specified font size, when drawing the text.


\section{More Possibilities}

There are still a lot of things not covered in the given examples. One of these
I want to cover here, as it can be handy when making plots on a regular basis.

\subsection{Styles}

When having a certain plotting style in mind setting all the necessary
parameters can be repetitive and tedious. The \mpl offers a set of predefined
\fref{https://matplotlib.org/users/style\_sheets.html}{styles sheets} that can
be used either by calling \pl{plt.style.use()}, or via context managers that
are obtained by calling \pl{plt.style.context()}, both times with the name of
the style as argument. This will set all parameters defined in the style to the
specified values. When using context managers, the plotting commands outside of
the context will still use the default values.  Parameters that are still set
by hand will override those set in the style sheet.

It is also possible to define custom style sheets. For this one simply has to
define a style file with all values that are supposed to be changed. Possible
values that can be set here are those, that can also be found in the 
\pl{matplotlibrc}, the file containing all default parameters.

To use it one can either supply the url of the file to the \pl{use()} or
pl{context()} at \pl{$\sim$/.config/matplotlib/mpl\_configdir/stylelib} with a
name of the form \pl{'<name>.mplstyle'} and then simply pass the name to the
corresponding functions.


\section{Conclusion}

Even though a lot of things have not been covered here one can hopefully see
that the \mpl is a powerful tool for data visualization. It can easily be used
to get a quick glimpse of the data of interest, but also to make very detailed
plots that are publication ready.


\section{Code for \fig{gs}}
\label{sec:gs}

\inputminted[linenos, mathescape]{python}{grid_spec.py}

\bibliography{bib}{}
\bibliographystyle{plain}


\end{document}
