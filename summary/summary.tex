\documentclass[a4paper, 11pt, twocolumn]{article}

% PACKAGES

\usepackage{float}
\usepackage[english]{babel}
\usepackage{pgfplots}  % Nice plots
\usepackage[colorlinks=true]{hyperref}
\usepackage{minted}  % Code highlighting
\usepackage{xspace}  % Add trailing space if neccesary

% COMMANDS

% Show a code sample with the resulting plot side by side
% Works well for code that has a max of 42 characters per line
\newcommand{\plotsample}[6][12cm]{
\begin{figure*}[h]
    
    \vspace{#2}
    \hspace{-1cm}
    \begin{minipage}[c][#1][t]{#3\textwidth} 
    \inputminted{python}{#4.py}
    \end{minipage}
    \hspace{0.5cm}
    \begin{minipage}[c][#1][t]{0.6\textwidth}
    \IfFileExists{#4.pgf}{
        \input{#4.pgf}
    }{
        \includegraphics[width=\linewidth]{#4.png}
    }
    \end{minipage}

\caption{#5}
\label{smpl:#6}
\end{figure*}
}

\newcommand{\mpl}{\texttt{matplotlib}\xspace}
\newcommand{\pl}[1]{\mbox{\texttt{#1}\xspace}}
\newcommand{\latex}{\LaTeX\xspace}

\title{matplotlib et al.}
\author{Johannes Haux}
\date{\today}

%DOCUMENT
\begin{document}

\maketitle

% \plotsample[8cm]{-1cm}{0.5}{axis_labels}{Adding labels to the axis is done with
%                                         two lines of code.}{lb}

When it comes to presenting data, \mpl is a very interesting
tool, that combines a lot of positive features:
It is 
free,
well documented with a lot of examples and
customizable to a very fine grained level.

This document accompanies a talk given in the seminar "Advanced Tools" during
the summer semester 2017. Here I will take a look at some basic concepts of
\mpl and then demonstrate some features using an example plot.


\section{A first Plot}

\begin{figure}[h!]
\centering
\input{axis_labels.pgf}
\caption{Plot generated with the code shown in listing \ref{l:al}}
\label{fig:al}
\end{figure}

\begin{listing}[h!]
\inputminted{python}{axis_labels.py}
\caption{Code for figure \ref{fig:al}.}
\label{l:al}
\end{listing}

To show how quickly one can make plots with the \mpl an example
is given in \mbox{figure \ref{fig:al}}.

Given an artificially created dataset \pl{data} we want to take a look at its
histogram.
To do this only one line of code is used to generate the plot: 
\pl{plt.hist(data)}
Using the commands \pl{plt.[x,y]label} one can quickly add labels to the
axis to make the plot easier to interpret.

Finally the plot is saved to using the \pl{plt.savefig(...)} command and shown
to the user with the \pl{plt.show()} command.

To ensure that the axis labels are visible when the plot is saved the method
\pl{f.tight\_layout()} is called on the figure object containing the plot.

Thus it is very easy to get a visual feedback of how your data looks like with
only a small number of lines of code! But of course when thinking about publishing, one wants to have more intricate customizability. This is absolutely 
possible using the \mpl, as can be seen for example in figure \ref{fig:gs}.
To understand what is happening there, we will first take a look at what is 
inside the plots generated with the \mpl.


\section{Components of a Plot}

A \mpl plot consists of 4 main components:
\begin{description}
\item[Figure]
The figure manages all axes in a plot and contains information on the size of
the plot. It can be instanciated by calling \pl{plt.figure()}. 
The current figure can be obained by calling \pl{plt.gcf()}.

\item[Axes]
The axes is where the actual plot lives. Here the data is drawn
as specified in the plot calls. To instanciate an axes given a figure \pl{f}
one can use \pl{f.add\_subplot(...)}. To generate both figure and axes it
is encouraged to use \pl{f, ax = plt.subplots(rows, cols)}.
The current axes can be obained by calling \pl{plt.gca()}.

\item[Axis]
Each axes contains a set of axis, which manage the data range that is to be
plotted as well as all ticks and ticklabels.

\item[Artist]
All components of a plot are drawn by artists when the plot is rendered.
E.g. a figure is also an artist, as well as axes and axis.
\end{description}

The \mpl has a notion of a current figure and axis. This allows
the user to make several plotting calls without explicitly specifying where
to draw the plot. As said above, these can be accessed by calling 
\pl{plt.gcf()} and \pl{plt.gca()} for current figure and axes respectively.


\section{Working with \mpl}

At times the many possibilities the \mpl offers can be overwhelming. There are 
nevertheless ways to help yourself.

\begin{description}
\item[Examples]
If one is unsure what kind of plot one wants to make or has a rough idea, what
it should look like, the \href{https://matplotlib.org/examples/}{examples} page
of the matplotlib documentation is very helpful.
\item[Documentation]
When in doubt how to use a certain object or method, the
\mpl \href{https://matplotlib.org/contents.html}{documentation} covers it! 
\item[StackOverflow]
If the above two ways do not help, it might be useful to note, that there is a
very active user community, helping each other out, e.g. on
\href{https://stackoverflow.com/questions/tagged/matplotlib}{stackoverflow}.
Most solutions to most problems can be found there.
\end{description}



\section{A more complex Plot}

\label{sec:plt}

\begin{figure*}[ht!]
\centering
\input{grid_spec.pgf}
\caption{Complex plot example showing different plot types and ways to
         structure the appearance. It is discussed in detail in section
         \ref{sec:plt}.}
\label{fig:gs}
\end{figure*}

In the following, whenever line numbers are referenced, these are taken from
section \ref{sec:gs}, unless otherwise specified.

In figure \ref{fig:gs} one can find the plot that is generated by the code
found in section \ref{sec:gs}. Again artificially generated data is used.

When comparing the code shown in listing \ref{l:al} and section \ref{sec:gs}
respectively, one can first of all notice several differences in how the plots
are created.

\subsection{Coding Style}
The code shown in figure \ref{fig:al} is written in the "pyplot" coding style,
also called "MATLAB" coding style, as it tries to emulate the plotting commands
used in MATLAB.

This coding style can be used to quickly create simple plots, yet it can
grow unreadbly very fast if the plot becomes more and more complex.

The alternative to this coding style is the object oriented approach, which 
uses explicit handles for figures and axes objects. It has the great advantage,
that it make s code easier to read and allows access to more methods, that 
would be harder to use when using pure pyplot coding.


\subsection{Multiple Axes}

The simple plot consisted only of one axes and figure, that were both
implicitly created with the \pl{plt.hist()} call. In the second example we find
several axes in the figure, which are created explicitly.

There are several ways to draw multiple axes in one figure. When wanting to
draw a regular grid of plots the easiest method would be
\mint{python}|fig, AX = plt.subplots(nrows, ncols)|
which returns the figure and a list of all axes. The arguments \pl{nrows} and 
\pl{ncols} define the numbers of rows and columns of axes in the figure.

To get more control over the way the axes are drawn one can use the
\pl{GridSpec} object as is done in the example plot. It also defines a grid,
but in contrast to the approach above one can also define axes that span
several rows or columns or both, by passing slices of the object to
\pl{figure.add\_subplot()}. This is done for example in line \ref{gsslice}.
Additionaly one can also specify the height or width ratios of the axes using
the \pl{height\_ratios} or \pl{width\_ratios} keyword argument. The latter can
be seen in line \ref{gswr}. Of course the customizability does not end here,
more information on the
\href{https://matplotlib.org/users/gridspec.html}{GridSpec} object can be found
in the matplotlib documentation.


\subsection{Plot types}

We can also see a new plottype in figure \ref{fig:gs}: The lineplot, generated
with the \pl{ax.plot()} method (see line \ref{plot}).

There are very many different plot types \mpl supports, a look at the
\href{https://matplotlib.org/api/axes_api.html#plotting}{documentation} can
give an idea of how these look like.

Also note that when plotting the histogram in line \ref{hist} and below, some
keyword arguments are passed to draw a black edge around the bars and make them
translucent, such that the overlapping parts are all visible. Most plotting
methods support these and similar arguments, like \pl{linestyle}, \pl{color},
\pl{marker} or \pl{markersize}, which allow to explicitly specify the way the
plots are drawn.

Calling several plotting methods on one axis in succession without specifing the color explicitly will not result in multiple plots of the same color, but as
can be seen in figure \ref{fig:gs}, in plots of different colors. This allows
to easily stack plots in one axes without loosing the overview.

\subsection{Legends}

To know which plot corresponds to which data one can pass the \pl{label} 
keyword to each plotting method and call the \pl{ax.legend()} method on the
axes of interest. The draws a legend with the specified labels in a place on
the axes that covers as little of the plot as possible. This can be seen in
line \ref{leg}. The position of the legend can also be explicitly set by 
passing a string (e.g. \pl{'upper left'}, \pl{'center'}) or an int (e.g.
\pl{2}, \pl{10} respectively).

Calling \pl{legend()} without argurments will only take labels from plotting
calls on this axes into account. If one would like to make a legend inside a
different axes than the one where the data is plotted, one has to pass a list
of handles and a list of corresponding names to this method. The handles are
returned by the plotting methods.

Again a look at the \href{https://matplotlib.org/api/_as_gen/matplotlib.axes.Axes.legend.html#matplotlib.axes.Axes.legend}{documentation}
can be very helpful.


\subsection{Text and Annotation}

The most glaring difference between the simple and the complex plot are, that 
the in the latter the fonts are set in a latex like serif font. This is
accomplished by telling the \mpl to update some default parameters stored
in the \pl{rcParams} dictionary. To simply get use latex text, one has to set
the \pl{'text.usetex'} parameter to True and to use serif fonts, one has to
choose the \pl{'font.family'} accordingly. All text will now look like in a
\latex document.

But this is not all we can do: If one would like to draw a \latex formula in
the plot this is now possible. In the example this is done in line \ref{ann}
using the \href{https://matplotlib.org/users/annotations.html}{\pl{annotate()}}
method, but one could also use the
\href{https://matplotlib.org/api/pyplot_api.html#matplotlib.pyplot.text}
{\pl{text()}} method.

Please note that the string containing the text is a raw string. Thus no 
characters are escaped, which helps a lot when using expressions like
\pl{\\frac\{\}\{\}}, that would otherwise result in escaping \pl{\\f} to a
formfeed or pagebreak, something latex does not understand.

While \pl{text} simply draws the text where we want it, \pl{annotate} also
allows to draw an arrow between text and a point \pl{xy}, we have to specify.
The properties of the arrow are controlled using the \pl{arrow\_props}
dictionary. Here we chose to draw an arc with a simple arrowhead on the front,
but many other possibilities exist.

The coordinates that are specified are also interesting to look at. When
drawing text the default behaviour of \mpl is to interpret the coordinates like
data points. In our example this is the case for the \pl{xy} coordinates where
the arrow points. As the text is supposed to be place in the top left corner,
the coordinates \pl{xytext} here are chosen to be a fraction of the axes as
specified by passing the \pl{textcoords} argument. (0, 0) corresponds to the
bottom left corner of the axes, (1, 1) to the top right. Values outside the
$[0, 1]^2$ are outside of the axes.

To ensure that the text is not drawn over the borders of the plot we can
specify its alignment with the \pl{va} and \pl{ha} keywords for vertical and
horizontal alignment respectively. In our example they are set such that the 
top right of the bounding box of the text is at the corrdinates specified for 
the text.

Another example where text is modified can be found in line \ref{lbc}.
Sometimes it can be useful to explicitely position the labels of an axis. In
our example the y labels of the top left and bottom axes were not aligned, so
they were modeved using the \pl{ax.yaxis.set\_label\_coords()} method. Its
arguments are the x and y coordinates of the label as axes fraction at which
the text should be placed, just like above in the \pl{annotate} example.


\subsection{Exporting}
When we are done with tweaking the plot it is time to save it to a file, that
can later be reused. This is simply done by calling the \pl{f.save()} method 
on the figure object. The only thing that needs to be specified is name of the
file with the file ending corresponding the desired format. This could for 
example be \pl{.png}, \pl{.jpg}, \pl{.svg} or \pl{.pgf}.

The latter is of special interest to \latex users, as it saves the plot as a
tikz picture, that then can be placed inside a \latex document using the
command \pl{\\input\{<plot>.pgf\}}

The dimensions of the plot are those specified when instanciating the figure
using the \pl{figsize} argument. Please note that the unit of these values is
inches. 

It is also important to note, that the size of text in the plots corresponds
to the specified fontsize, when drawing the text.


\section{More Possibilities}

There are still a lot of things not covered in the given examples. Two of these
I want to cover here, as they can be handy at times.

\subsection{Drawing Geometry}

While there are a lot of predefined plot types, it can occur that one wants to
draw something very specifically, not covered with these. For this \mpl allows
to draw a set of geometric objects like Rectangles, Circles or Polygones.


\subsection{Styles}

When having a certain plotting style in mind setting all the neccessary
parameters can be repetitive and tedious. The \mpl offers a set of predefined
\href{https://matplotlib.org/users/style\_sheets.html}{styles sheets} that can
be used either by calling \pl{plt.style.use()}, or via context managers that
are obtained by calling \pl{plt.style.context()}, both times with the name of
the style as argument. This will set all parameters defined in the style to the
specified values. When using context managers, the plotting commands outside of
the context will still use the default values.  Parameters that are still set
by hand will override those set in the style sheet.

It is also possible to define custom style sheets. For this one simply has to
define a style file with all values that are supposed to be changed. Possible
values that can be set here are those, that can also be found in the 
\pl{matplotlibrc}, the file containing all default parameters.

To use it one can either supply the url of the file to the \pl{use()} or
pl{context()} at \pl{$\sim$/.config/matplotlib/mpl\_configdir/stylelib} with a
name of the form \pl{'<name>.mplstyle'} and then simply pass the name to the
corresponding functions.


\section{Other libraries}

\onecolumn

\section{Code for figure \ref{fig:gs}}
\label{sec:gs}

% \vspace{-9cm}
\inputminted[linenos, mathescape]{python}{grid_spec.py}

\end{document}
